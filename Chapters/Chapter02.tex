% Chapter 2
\newcommand{\systemname}[0]{VideoNote}

\chapter{Navigating Lecture Videos} % Chapter title

\label{ch:visualtranscript} % For referencing the chapter elsewhere, use

%----------------------------------------------------------------------------------------
How viewers watch videos depends on the content of the video and the viewers'
needs. For example, watching a film in a theater is a very different experience
from watching a video tutorial on YouTube about how to cook brussel sprouts.
Even for a same video, someone who is watching it for the first time may have a
different approach from someone who has seen the video previously and is
watching again only to review. In this chapter, we focus on the scenario of
watching lecture videos. \\

Lecture videos are growing in popularity through Massive Open Online Courses
(MOOCs) and flipped classroom models.
However, learning with these videos using existing video player interfaces can
be challenging. Viewers cannot digest the lecture material at their
own pace, and it is also difficult to search or skim the content. For these and other
reasons, some viewers prefer lecture notes or textbooks to videos.\\

To address these limitations, we design \textbf{\systemname}, a readable
interface for lecture videos that resmebles lecture notes with figures and text.
\systemname\ combines visuals presented in the lecture
with its transcript text. To generate a \systemname, we first segment the visual content
of a lecture into a set of discrete images that correspond to equations, figures, or lines of text. Then, we analyze
the temporal correspondence between the transcript and the visuals to determine their relationships. Finally, based on the inferred relationships, we arrange the text and visuals into a hierarchical and linear layout. \\

We compare our interface with a standard video player, and a state-of-the-art interface designed specifically for 
lecture videos. User evaluation suggests that users prefer \systemname\ for the task of learning and that \systemname\
facilitates browsing and searching in lecture videos.

%----------------------------------------------------------------------------------------

\section{Introduction}
Despite the increasingly important and broad role of lecture videos in education, learning from such videos poses some challenges. 
%
It is difficult for viewers to consume video content at their own pace~\cite{chi2012mixt}.
%
To skip quickly through familiar concepts or slowly review more difficult material, the viewer must interrupt playback and scrub back-and-forth in the timeline.
%
It is also difficult to find specific information in a video. While scrubbing allows users to browse the visual information in the lecture, it is not effective for skimming the audio content, which often includes critical explanations and context that accompany the visuals. As an alternative, some platforms (e.g., Khan Academy and YouTube) provide synchronized transcripts that allow users to click on a phrase and play the video at that location. However, skimming the transcript for relevant content can also be challenging since the text is not structured, and viewers must click on various parts of the text to see the corresponding visuals. 
%
Finally, it is hard to get a quick overview of the lecture content without watching the entire video. 
For these and other reasons, some people prefer static learning materials such as textbooks or printed lecture notes over videos.\\

Inspired by lecture notes, we present \systemname, a readable interface for both the visual and audio content of a lecture video that facilitates reviewing, browsing and navigation. 
%
We focus on blackboard-style lectures that show a (possibly infinite) blackboard where the instructor writes down by hand the content of the lecture. \systemname s aggregate the full lecture content in a structured format where visual information is segmented and grouped with the corresponding narration text. For example, Figure~\ref{Fig:teaser}(b) shows our automatically generated output for a math lecture that interleaves verbal explanations with the corresponding equations written on the board. 
%
By default, \systemname s hide redundant information to show a compact representation of the content that viewers can expand interactively to show relevant details (Figure~\ref{Fig:teaser}(c)). 
%
Presenting video content in this manner allows users to review the lecture at their own pace while getting both the visual and textual information in a readable, skimmable format.
%
\systenmame s is also linked to the video such that clicking on the text or the visuals plays the video from the corresponding location.
%
In this respect, \systemname s offer many of the benefits of traditional static media, such as textbooks and lecture notes, while also giving viewers direct access to the video content.\\

There are two main challenges in transforming a video and its transcribed audio into a \systemname : (1) visuals, which are drawn progressively on the board, must be discretized into  a set of meaningful figures, and (2) such figures and text representing the audio content must be organized into a compact, structured format that emphasizes the relationships between the two channels of information.
%
To segment the visuals into meaningful figures, we propose a dynamic programming approach that takes into account both the spatial layout of strokes and the time when they were drawn. We further time-align the transcript with the audio and use this alignment to establish correspondences between the visuals and the text. Finally, we use the visual-text correspondence to detect redundant information and arrange the content in a compact, sequential layout where the text is organized into readable paragraphs.\\

We evaluate our approach with a user study that compares \systemname\ with a baseline transcript-based video player, and an existing, state-of-the-art visual-based video player developed by Monserrat et al \cite{monserrat2013notevideo}. We measure performance on summarization and search tasks, and observe how the participants interact with the interfaces. We find that \systemname\ is an effective interface for studying lecture videos. Specifically, users performed best using \systemname\ for search tasks involving text. Users noted that \systemname\ helped them to get a quick overview of the video including the details conveyed only through the text, and to efficiently focus on parts of interest. They also found the structured text easier to read and connect to relevant visuals than the baseline text-only transcript. In a post-study survey, users strongly preferred our interface for learning over the other two interfaces.

\section{Previous Work}
\section{Design Principles}
\section{Algorithms}
\section{User Evaluation}
\section{Discussion}